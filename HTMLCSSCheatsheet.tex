%
%  handleidinghuis
%
%  Created by Niels Doorn on 2010-07-04.
%  Copyright (c) 2010 __MyCompanyName__. All rights reserved.
%
\documentclass[8pt,pagesize,footinclude=false,headinclude=false]{scrartcl}


% Copyright (C) 2008, 2009, 2010 Bert Burgemeister
%
% Permission is granted to copy, distribute and/or modify this
% document under the terms of the GNU Free Documentation License,
% Version 1.2 or any later version published by the Free Software
% Foundation; with no Invariant Sections, no Front-Cover Texts and
% no Back-Cover Texts. For details see file COPYING.
%
\usepackage{amsmath}
\usepackage{amsfonts}
\usepackage{amssymb}
\usepackage{rotating}
\usepackage{graphicx}
\usepackage{multicol}
\usepackage{textcase}
\usepackage{textcomp}
\usepackage{ulem}
\usepackage{suffix}
\usepackage{makeidx}
\usepackage[pagestyles]{titlesec}
\usepackage{titletoc}
\usepackage{xcolor}
%
%%%%%%%%%%%%%%%%%%
% Two font alternatives:
% (A) All (except cover pages) Computer Modern --
%     everything comes from the same sound root; gets about 5% longer
%     than alternative (B) 
\usepackage{type1cm}
\usepackage{exscale}
%%%%%%%%%%%%%%%%%%
% (B) Times mixed with Helvetica --
%     different sources; need scaling as they don't even agree in
%     their concept of height
%\usepackage{mathptmx}
%\usepackage[scaled]{helvet}
%%%%%%%%%%%%%%%%%%
%


% outsourced page dimensions for A4 paper
%\setlength{\paperwidth}{10.5cm}
%\setlength{\paperheight}{29.7cm}
%%\areaset[10mm]{8cm}{29cm}
%\typearea[3mm]{20}

% Use utf-8 encoding for foreign characters
\usepackage[utf8]{inputenc}

\usepackage[dutch]{babel}

\usepackage[a4paper,landscape]{geometry}
%\usepackage[a4paper,hmargin=1cm,vmargin=1cm]{geometry}
\usepackage{multicol}

% Setup for fullpage use
\usepackage[cm]{fullpage}

\usepackage{eurosym}

% Surround parts of graphics with box
\usepackage{boxedminipage}

% Package for including code in the document
\usepackage{listings}

% If you want to generate a toc for each chapter (use with book)
\usepackage{minitoc}

% This is now the recommended way for checking for PDFLaTeX:
\usepackage{ifpdf}

\definecolor{lightgray}{rgb}{0.01,0.01,0.01}
\usepackage{color}
\definecolor{lightgray}{rgb}{0.95, 0.95, 0.95}
\definecolor{darkgray}{rgb}{0.4, 0.4, 0.4}
%\definecolor{purple}{rgb}{0.65, 0.12, 0.82}
\definecolor{editorGray}{rgb}{0.95, 0.95, 0.95}
\definecolor{editorOcher}{rgb}{1, 0.5, 0} % #FF7F00 -> rgb(239, 169, 0)
\definecolor{editorGreen}{rgb}{0, 0.5, 0} % #007C00 -> rgb(0, 124, 0)
\definecolor{orange}{rgb}{1,0.45,0.13}		
\definecolor{olive}{rgb}{0.17,0.59,0.20}
\definecolor{brown}{rgb}{0.69,0.31,0.31}
\definecolor{purple}{rgb}{0.38,0.18,0.81}
\definecolor{lightblue}{rgb}{0.1,0.57,0.7}
\definecolor{lightred}{rgb}{1,0.4,0.5}
\usepackage{upquote}

\title{HTML en CSS Cheatsheet}
\author{Niels Doorn}
\date{\today}

\usepackage[bookmarks=true,pdftex,bookmarksopen=true,bookmarksnumbered=true,pdfborder={0 0 0 0}]{hyperref}


\titleformat{\section}{\sffamily\mdseries\slshape}
            {\huge\thesection}{.7em}{\huge}[{\titlerule[0.25pt]}]
            
\titleformat{\subsection}{\sffamily\mdseries\slshape}
            {\Large\thesubsection}{.7em}{\Large}[{\titlerule[0.25pt]}]

\lstdefinelanguage{JavaScript}{
  basicstyle=\footnotesize\ttfamily,
  keywords={typeof, new, true, false, catch, function, return, null, catch, switch, for, Array, var, const, if, in, while, do, else, case, break},
  keywordstyle=\color{blue}\bfseries,
  ndkeywords={class, export, boolean, throw, implements, import, this, querySelector, innerHTML, style, window, document},
  ndkeywordstyle=\color{darkgray}\bfseries,
  identifierstyle=\color{black},
  sensitive=false,
  comment=[l]{//},
  morecomment=[s]{/*}{*/},
  commentstyle=\color{purple}\ttfamily,
  stringstyle=\color{red}\ttfamily,
  morestring=[b]',
  morestring=[b]"
}

% CSS
\lstdefinelanguage{CSS}{
  keywords={color,background-image:,background-color:,margin:,padding:,font:,weight:,display:,position:,top,left,right,bottom,list,style,border:,size,white,space,min,width:, transition:, height:, transform:, transition-property:, transition-duration:, transition-timing-function, background-color:, margin-top:, margin-bottom:, margin-left:, margin-right:,list-style:, text-decoration:, text-decoration:, background-repeat:, box-sizing:, -webkit-box-sizing:, -moz-box-sizing:, color:, padding-top:, padding-bottom:,padding-left:,padding-right:,font-size:},	
  sensitive=true,
  morecomment=[l]{//},
  morecomment=[s]{/*}{*/},
  morestring=[b]',
  morestring=[b]",
  alsoletter={:},
  alsodigit={-}
}

\lstdefinelanguage{HTML5}{
  language=html,
  sensitive=true,	
  alsoletter={<>=-},	
  morecomment=[s]{<!-}{-->},
  tag=[s],
  otherkeywords={
  % General
  >,
  % Standard tags
	<!DOCTYPE,
	</b, </i, </em, </strong, </q, </blockquote, </sup, </sub, <b, <i, <em, <strong, <q, <blockquote, <sup, <sub,
  </html, <html, </body, <body, <head, <title, </title, <style, </style, <link, </head, <meta, />,
	% body
	</body, <body,
	% Divs
	</div, <div, </div>, 
	% Paragraphs
	</p, <p, </p>, </hr, </br, <hr, <br,
	%lists
	</ul, </li, <li, <ul, <link, </ol, <ol, </dt, </dl, </dd, <dt, <dl, <dd,
	% headings
	</h1, </h2, </h3, </h4, </h5, </h6, <h1, <h2, <h3, <h4, <h5, <h6,
	% forms
	</form, </input, </textarea, </label, </input, <form, <input, <textarea, <label,
	% opmaak
	% scripts
	</script, <script,
  % More tags...
  <canvas, /canvas>, <svg, <rect, <animateTransform, </rect>, </svg>, <video, <source, <iframe, </iframe>, </video>, <image, </image>, <header, </header, <article, </article
  },
  ndkeywords={
  % General
  =,
  % HTML attributes
  charset=, src=, id=, width=, height=, style=, type=, rel=, href=,
  % SVG attributes
  fill=, attributeName=, begin=, dur=, from=, to=, poster=, controls=, x=, y=, repeatCount=, xlink:href=,
  % properties
  margin:, padding:, background-image:, border:, top:, left:, position:, width:, height:, margin-top:, margin-bottom:, font-size:, line-height:,
	% CSS3 properties
  transform:, -moz-transform:, -webkit-transform:,
  animation:, -webkit-animation:,
  transition:,  transition-duration:, transition-property:, transition-timing-function:,
  }
}

\lstdefinestyle{htmlcssjs} {%
  % General design
%  backgroundcolor=\color{editorGray},
  basicstyle={\footnotesize\ttfamily},   
  frame=b,
  % line-numbers
  xleftmargin={0.75cm},
  numbers=left,
  stepnumber=1,
  firstnumber=1,
  numberfirstline=true,	
  % Code design
  identifierstyle=\color{black},
  keywordstyle=\color{blue}\bfseries,
  ndkeywordstyle=\color{editorGreen}\bfseries,
  stringstyle=\color{editorOcher}\ttfamily,
  commentstyle=\color{brown}\ttfamily,
  % Code
  language=HTML5,
  alsolanguage=JavaScript,
  alsodigit={.:;},	
  tabsize=2,
  showtabs=false,
  showspaces=false,
  showstringspaces=false,
  extendedchars=true,
  breaklines=true,
  % German umlauts
  literate=%
  {Ö}{{\"O}}1
  {Ä}{{\"A}}1
  {Ü}{{\"U}}1
  {ß}{{\ss}}1
  {ü}{{\"u}}1
  {ä}{{\"a}}1
  {ö}{{\"o}}1
}


\lstloadlanguages{% Check Dokumentation for further languages ...
	%[Visual]Basic
	%Pascal
  %C
	%C++
	%XML,
	%HTML,
	%Java,
	%PHP,
	HTML5,
	CSS
}


\lstset{
        basicstyle=\footnotesize\ttfamily, % Standardschrift
        %numbers=left,               % Ort der Zeilennummern
        numberstyle=\tiny,          % Stil der Zeilennummern
        %stepnumber=2,               % Abstand zwischen den Zeilennummern
        numbersep=5pt,              % Abstand der Nummern zum Text
        tabsize=2,                  % Groesse von Tabs
        extendedchars=true,         %
        breaklines=true,            % Zeilen werden Umgebrochen
        %keywordstyle=\bfseries,
		%commentstyle=\em\color{gray},
        frame=trlb,         
%        keywordstyle=[1]\textbf,    % Stil der Keywords
%        keywordstyle=[2]\textbf,    %
%        keywordstyle=[3]\textbf,    %
%        keywordstyle=[4]\textbf,   \sqrt{\sqrt{}} %
%        stringstyle=\bfseries\color{gray}, % Farbe der String
        showspaces=false,           % Leerzeichen anzeigen ?
        showtabs=false,             % Tabs anzeigen ?
        xleftmargin=0pt,
        framexleftmargin=0pt,
        framexrightmargin=0pt,
        framexbottommargin=0pt,
        %backgroundcolor=\color{lightgray},
			  %stringstyle=\color{black},
			  %identifierstyle=\color{blue},
			  %keywordstyle=\color{cyan},
        showstringspaces=false,      % Leerzeichen in Strings anzeigen ?
			  % Code design
			  identifierstyle=\color{black},
			  keywordstyle=\color{blue}\bfseries,
			  ndkeywordstyle=\color{editorGreen}\bfseries,
			  stringstyle=\color{editorOcher}\ttfamily,
			  commentstyle=\color{brown}\ttfamily,
			  % Code
			  language=HTML5,
			  alsolanguage=JavaScript,
			  alsodigit={.:;},	
			  tabsize=2,
			  showtabs=false,
			  showspaces=false,
			  showstringspaces=false,
			  extendedchars=true,
			  breaklines=true,
				upquote=true
}


\hypersetup{
	pdfauthor = {Niels Doorn},
	pdftitle = {HTML en CSS Cheatsheet},
	pdfsubject = {HTML, CSS},
	pdfkeywords = {HTML,css,cheatsheet},
	pdfcreator = {NielsDoorn/RocVanTwente}
}

\usepackage{lastpage}
\usepackage{fancyhdr}
\pagestyle{fancy}
\rhead{}
\lhead{}
\chead{}
\lfoot{Versie \today - Niels Doorn \copyright~\the\year}
\cfoot{\url{http://www.nielsdoorn.nl}}
\rfoot{\thepage\ van \pageref{LastPage}}
\renewcommand{\headrulewidth}{0pt}
\renewcommand{\footrulewidth}{0pt}

\usepackage{pst-barcode}

\begin{document}

\ifpdf
\DeclareGraphicsExtensions{.pdf, .jpg, .tif}
\else
\DeclareGraphicsExtensions{.eps, .jpg}
\fi

\begin{multicols*}{4}

\section*{HTML \& CSS Cheatsheet}
Deze cheatsheet bevat de meest gebruikte HTML tags en CSS eigenschappen voor het vak webdesign.

\section*{HTML}
HTML vormt de structuur en de inhoud van een pagina en bestaat uit tags die door de browser worden begrepen. Iedere tag heeft zo z'n eigen specifieke eigenschappen en functionaliteit. Een tag staat tussen scherpe haken \texttt{< >} bijvoorbeeld een \texttt{<div>} tag. Een aantal tags komen altijd voor in een pagina en sommige tags gebruik je alleen als je ze nodig hebt. Iedere tag die je opent moet je ook weer afsluiten. Dat kan op twee manieren:
\begin{lstlisting}[language=HTML5]
	<!-- openen en later sluiten -->
	<div> </div>
	<!-- openen en direct sluiten -->
	<br />
\end{lstlisting}
\subsection*{Minimale HTML pagina}
Hieronder een minimale pagina in HTML, deze structuur heb je altijd nodig.
\lstinputlisting{code/minimal.html}
HTML tag: HTML document\\
HEAD tag: Informatie over de pagina\\
BODY tag: Inhoud van de pagina

\subsection*{Een CSS stylesheet koppelen}
Aan de HTML pagina kan je een CSS stylesheet koppelen voor de vormgeving.\\
\noindent Plaats binnen de HEAD tag:
\begin{lstlisting}[language=HTML5]
	<link href="style.css" rel="stylesheet" />
\end{lstlisting}
\noindent Gebruik in principe \underline{nooit} inline CSS maar \underline{altijd} een aparte stylesheet in een apart bestand.

\subsection*{Een JavaScript bestand koppelen}
Aan de HTML pagina kan je een of meerdere JavaScript bestanden koppelen.\\
\noindent Plaats binnen de HTML tag:
\begin{lstlisting}[language=HTML5]
	<script src="mijnScript.js" />
\end{lstlisting}
\noindent Gebruik zoveel mogelijk JavaScript in een aparte bestanden.

\subsection*{Commentaar}
Commentaar in HTML:
\begin{lstlisting}[language=HTML5]
<!--
commentaar
-->
\end{lstlisting}

\subsection*{Tekst}
Headings:
\begin{lstlisting}[language=HTML5]
<h1>Titel niveau 1</h1>
<h2>Titel niveau 2</h2>
...
<h6>Titel niveau 6</h6>
\end{lstlisting}

\noindent Tekstopmaak:
\begin{lstlisting}[language=HTML5]
<b>Vetgedruk</b>
<i>Schuingedrukt</i>
<strong>Belangrijk</strong>
<em>Benadrukken</em>
E=MC<sup>2</sup> superscript
CO<sub>2</sub> subscript
<blockquote>Lang citaat</blockquote>
<q>Kort citaat</q> (werkt niet in IE)
\end{lstlisting}

\noindent Tekstindeling:
\begin{lstlisting}[language=HTML5]
<p>Paragraaf</p>
<br /> Nieuwe regel
<hr /> Horizontale lijn
\end{lstlisting}

\subsection*{Pagina-indeling}
Div'jes gebruik je om de inhoud van je pagina op te delen in gebieden. Vervolgens kun je die met CSS een plaats, afmeting en andere eigenschappen geven. 
\begin{lstlisting}[language=HTML5]
<div>Een gebied op een pagina, bijvoorbeeld een kolom</div>
\end{lstlisting}

\subsection*{Afbeeldingen}
Een afbeelding in HTML: het \textbf{src} attribuut verwijst naar de afbeelding die je wilt laten zien.
\begin{lstlisting}[language=HTML5]
<img src="Niels.jpg" width="100" height="100" alt="Portret van een webdesigner" title="Niels"/>
\end{lstlisting}

\subsection*{Links}
Linkjes werken op de volgende manier. Eerst geef je in de A tag aan naar welke pagina je wilt linken met het href attribuut. Daarna plaats je tussen de begin en eind tag \textbf{tekst} of een \textbf{afbeelding} waar je op wilt klikken. In de onderstaande code is dat `Contactpagina'.

\begin{lstlisting}[language=HTML5]
<a href="contact.html">Contactpagina</a>
\end{lstlisting}

\subsection*{Lijstjes}
In HTML zijn er twee belangrijke soorten lijstjes. Lijstjes die genummerd zijn (ge-ordend) en lijstjes met bullets (ongeordend).

\subsubsection*{Ongeordende lijsten}
Unordered list (boodschappenlijstje): 
\begin{lstlisting}[language=HTML5]
<ul>
	<li>Banaan</li>
	<li>Aardbei</li>
	<li>Appel</li>
</ul>
\end{lstlisting}

\subsubsection*{Genummerde lijsten}
Een ordered list (grootste steden ter wereld):
\begin{lstlisting}[language=HTML5]
<ol>
	<li>Shanghai</li>
	<li>Bombay</li>
	<li>Karachi</li>
	<li>Istanboel</li>
	<li>Delhi</li>
</ol>
\end{lstlisting}

\subsubsection*{Definitie lijsten}
Minder vaak gebruikt maar ook handig is de lijst met definities:
\begin{lstlisting}[language=HTML5]
<dl>
	<dt>Koffie</dt>
		<dd>Het zwarte goud, onmisbaar voor goed functioneren docenten</dd>
	<dt>Melk</dt>
		<dd>De witte motor</dd>
</dl>
\end{lstlisting}
\noindent De DT tag geeft een woord waarvan je de definitie wilt geven (koffie). De daarop volgende DD tag geeft de definitie.

\subsubsection*{Lijsten met meeerdere niveaus}
Je kunt verschillende niveaus aanbrengen in lijstjes.

\begin{lstlisting}[language=HTML5]
<ul>
  <li>Koffie</li>
  <li>Thee
    <ul>
    <li>Zwarte thee</li>
    <li>Groene thee</li>
    </ul>
  </li>
  <li>Melk</li>
</ul>
\end{lstlisting}

\subsection*{Formulieren}
Formulieren gebruik je om gegevens van de gebruiker naar een website te versturen. Een bezoeker kan bijvoorbeeld een contactformulier invullen en versturen via een website.

Een formulier bevat invulvelden met labels en knoppen. Een formulier kan met POST en met GET worden verstuurd naar een URL.

\begin{lstlisting}[language=HTML5]
<form action="informatieaanvraag.php" method="post">
	<label for="voornaam">Voornaam:</label>
	<input type="text" name="voornaam" />
	<label for="achternaam">Achternaam:</label>
	<input type="text" name="achternaam" />
	<label for="bericht">Uw vraag of bericht:</label>
	<textarea name="bericht"></textarea>
	<input type="submit" value="verzenden" />
</form>
\end{lstlisting}

\clearpage

\section*{CSS}
\subsection*{Begin van een CSS file}
\begin{lstlisting}[language=CSS]
@charset "utf-8";
\end{lstlisting}

\subsection*{Commentaar in CSS}
\begin{lstlisting}[language=CSS]
/* commentaar */
\end{lstlisting}

\subsection*{CSS selectoren}
In CSS selecteer je de HTML elementen die je wilt vormgeven en vervolgens geef je aan hoe die elementen vormgegeven moeten worden.

\noindent Selecteren op \underline{tag}
\begin{lstlisting}[language=CSS]
body {
}
div {
}
\end{lstlisting}

\noindent Selecteren op \underline{class} (een class mag meerdere keren voorkomen op een HTML pagina)
\begin{lstlisting}[language=CSS]
.menu-item {
}
\end{lstlisting}
\noindent Bijbehorende HTML code:
\begin{lstlisting}[language=HTML5]
<div class="menu-item"></div>
\end{lstlisting}

\noindent Selecteren op \underline{id} (een id mag slechts een keer voorkomen op een pagina)
\begin{lstlisting}[language=CSS]
#header {
}
\end{lstlisting}
\noindent Bijbehorende HTML code:
\begin{lstlisting}[language=CSS]
<div id="header"></div>
\end{lstlisting}

\noindent Selecteren op \underline{type en id} (een div met het id header)
\begin{lstlisting}[language=CSS]
div#header {
}
\end{lstlisting}

\noindent Selecteren op \underline{type en class} (alle divs met de klasse menu-item)
\begin{lstlisting}[language=CSS]
div.menu-item {
}
\end{lstlisting}

\subsection*{Kleuren}
Achtergrondkleur, font kleur
\begin{lstlisting}[language=CSS]
#inhoud {
	background-color: #0C9;
	color: white;
}
\end{lstlisting}

\subsection*{Afmetingen}
De afmetingen van de \underline{inhoud} van een element, dus niet de totale afmetingen. Totale afmeting = width + padding + border + margin.
In pixels:
\begin{lstlisting}[language=CSS]
#inhoud {
	width: 300px;
	height: 400px;
}
\end{lstlisting}

\noindent In procenten:
\begin{lstlisting}[language=CSS]
#inhoud {
	width: 50%;
	height: 20%;
}
\end{lstlisting}

\subsection*{Padding}
De ruimte die aan de \underline{binnenkant} wordt vrijgehouden.
\begin{lstlisting}[language=CSS]
#inhoud {
	/* alleen bovenaan 10px */
	padding-top: 10px; 

	/* of rondom 10px */
	padding: 10px; 

	/* of top, right, bottom, left */
	padding: 10px 0px 15px 0px; 
}
\end{lstlisting}

\subsection*{Margin}
De ruimte die aan de \underline{buitenkant} wordt vrijgehouden.
\begin{lstlisting}[language=CSS]
	/* alleen bovenaan 10px */
	margin-top: 10px; 
	
	/* of rondom 10px */
	margin: 10px; 
	
	/* of top, right, bottom, left */
	margin: 10px 0px 15px 0px; 
\end{lstlisting}

\subsection*{Borders}
Sommige elementen, zoals div'jes kunnen ook worden voorzien van een rand.
\begin{lstlisting}[language=CSS]
div#sidebar {
	border-width: 1px;
	border-style: dashed;
	border-color: black;
}
\end{lstlisting}

\noindent Of helemaal geen borders:

\begin{lstlisting}[language=CSS]
div#sidebar {
	border: none;
}
\end{lstlisting}

\subsection*{Box model border box}
Breedte inclusief padding, border en margin:
\begin{lstlisting}[language=CSS]
* { 
	-moz-box-sizing: border-box; 
	-webkit-box-sizing: border-box; 
	box-sizing: border-box;
}
\end{lstlisting}

\subsection*{Kolommen}
Meerdere kolommen naast elkaar
\begin{lstlisting}[language=HTML5]
<div id="wrap">
	<div id="inhoud"></div>
	<div id="sidebar"></div>
</div>
\end{lstlisting}

\begin{lstlisting}[language=CSS]
#wrap {
	display: table;
}

#inhoud {
	display: table-cell;
	width: 70%;
}

#sidebar {
	display: table-cell;
	width: 30%;
}
\end{lstlisting}

\subsection*{Een DIV centreren}
\begin{lstlisting}[language=CSS]
div#container {
	width: 850px;
	margin: 0 auto;
}
\end{lstlisting}

\subsection*{Achtergrondafbeelding}
Een achtergrond afbeelding op een element. Je kunt o.a. aangeven welke afbeelding en hoe de afbeelding herhaald moet worden.
\begin{lstlisting}[language=CSS]
div#header {
	background-image: url(achtergrond.jpg);
	background-repeat: none;
}
\end{lstlisting}

\subsection*{UL als menu stylen}
HTML met een UL als menu:
\begin{lstlisting}[language=HTML5]
<div id="menu"> 
  <ul>
	<li class="selected"><a href="index.html">Home</a></li>
    <li><a href="info.html">Info</a></li>
    <li><a href="contact.html">Contact</a></li>
  </ul>
</div>
\end{lstlisting}
CSS style:
\lstinputlisting[language=CSS]{code/menu.css}

\clearpage

\section*{Meer over CSS selectoren}
Een HTML pagina is een boomstructuur van tags. Je kunt zeggen dat iedere tag (behalve de HTML tag) een kind is van een andere tag. CSS biedt veel mogelijkheden om specifieke tags te selecteren op basis van deze structuur. Bijvoorbeeld alle h1 elementen binnen een div met een bepaalde class. Of alle oneven list items.

\subsection*{Selectie van childs en descendants}
Een \textbf{descendants} selector gebruik je voor het selecteren van alle elementen binnen een ander element (meerdere niveaus diep).

Een \textbf{child} selector gebruik je voor het selecteren van directe kinderen binnen een element (een niveau diep).\\
\noindent Voorbeeld HTML:
\begin{lstlisting}[language=HTML5]
	<div class="kolom">
		<h1>Agenda</h1>
		<div class="widget">
			<h1>Kalender</h1>
		</div>
	</div>
\end{lstlisting}

\noindent CSS:
\begin{lstlisting}[language=CSS]
/* child selector: selecteer alleen de agenda H1 in de kolom, niet de kalender H1 */
.kolom > h1 {
	color: blue;
}

/* descendants selector: selecteer alle h1 elementen binnen de kolom (agenda en kalender) */
.kolom h1 {
	font-size: 18pt;
}
\end{lstlisting}

\subsection*{Handige websites}
Er zijn veel sites met informatie over HTML en CSS, hier een paar voorbeelden:
\begin{itemize}
	\item Lorem Ipsum generator \url{http://bit.ly/3Ukgp}
	\item Dabblet \url{dabblet.com}
	\item CodePen \url{codepen.io}
	\item Mozilla Developers Network \url{http://goo.gl/UzgJQ}
	\item Moz. HTML reference \url{http://goo.gl/JRZ0P}
	\item Moz. CSS reference \url{http://goo.gl/EeWJe}
\end{itemize}

\end{multicols*}
\end{document}