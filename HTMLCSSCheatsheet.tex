%
%  handleidinghuis
%
%  Created by Niels Doorn on 2010-07-04.
%  Copyright (c) 2010 __MyCompanyName__. All rights reserved.
%
\documentclass[8pt,pagesize,footinclude=false,headinclude=false]{scrartcl}


% Copyright (C) 2008, 2009, 2010 Bert Burgemeister
%
% Permission is granted to copy, distribute and/or modify this
% document under the terms of the GNU Free Documentation License,
% Version 1.2 or any later version published by the Free Software
% Foundation; with no Invariant Sections, no Front-Cover Texts and
% no Back-Cover Texts. For details see file COPYING.
%
\usepackage{amsmath}
\usepackage{amsfonts}
\usepackage{amssymb}
\usepackage{rotating}
\usepackage{graphicx}
\usepackage{multicol}
\usepackage{textcase}
\usepackage{textcomp}
\usepackage{ulem}
\usepackage[usenames,dvips]{color}
\usepackage{suffix}
\usepackage{makeidx}
\usepackage[pagestyles]{titlesec}
\usepackage{titletoc}
%
%%%%%%%%%%%%%%%%%%
% Two font alternatives:
% (A) All (except cover pages) Computer Modern --
%     everything comes from the same sound root; gets about 5% longer
%     than alternative (B) 
\usepackage{type1cm}
\usepackage{exscale}
%%%%%%%%%%%%%%%%%%
% (B) Times mixed with Helvetica --
%     different sources; need scaling as they don't even agree in
%     their concept of height
%\usepackage{mathptmx}
%\usepackage[scaled]{helvet}
%%%%%%%%%%%%%%%%%%
%


% outsourced page dimensions for A4 paper
%\setlength{\paperwidth}{10.5cm}
%\setlength{\paperheight}{29.7cm}
%%\areaset[10mm]{8cm}{29cm}
%\typearea[3mm]{20}

% Use utf-8 encoding for foreign characters
\usepackage[utf8]{inputenc}

\usepackage[dutch]{babel}

\usepackage[a4paper,landscape]{geometry}
%\usepackage[a4paper,hmargin=1cm,vmargin=1cm]{geometry}
\usepackage{multicol}

% Setup for fullpage use
\usepackage[cm]{fullpage}

\usepackage{eurosym}

% Surround parts of graphics with box
\usepackage{boxedminipage}

% Package for including code in the document
\usepackage{listings}

% If you want to generate a toc for each chapter (use with book)
\usepackage{minitoc}

% This is now the recommended way for checking for PDFLaTeX:
\usepackage{ifpdf}

\definecolor{lightgray}{rgb}{0.01,0.01,0.01}

\title{HTML CSS Cheatsheet}
\author{Niels Doorn}
\date{\today}

\usepackage[bookmarks=true,pdftex,bookmarksopen=true,bookmarksnumbered=true,pdfborder={0 0 0 0}]{hyperref}


\titleformat{\section}{\sffamily\mdseries\slshape}
            {\huge\thesection}{.7em}{\huge}[{\titlerule[0.25pt]}]
            
\titleformat{\subsection}{\sffamily\mdseries\slshape}
            {\Large\thesubsection}{.7em}{\Large}[{\titlerule[0.25pt]}]


\lstset{
        basicstyle=\footnotesize\ttfamily, % Standardschrift
        %numbers=left,               % Ort der Zeilennummern
        numberstyle=\tiny,          % Stil der Zeilennummern
        %stepnumber=2,               % Abstand zwischen den Zeilennummern
        numbersep=5pt,              % Abstand der Nummern zum Text
        tabsize=2,                  % Groesse von Tabs
        extendedchars=true,         %
        breaklines=true,            % Zeilen werden Umgebrochen
        keywordstyle=\bfseries,
		%commentstyle=\em\color{gray},
        frame=trlb,         
%        keywordstyle=[1]\textbf,    % Stil der Keywords
%        keywordstyle=[2]\textbf,    %
%        keywordstyle=[3]\textbf,    %
%        keywordstyle=[4]\textbf,   \sqrt{\sqrt{}} %
        stringstyle=\bfseries\color{gray}, % Farbe der String
        showspaces=false,           % Leerzeichen anzeigen ?
        showtabs=false,             % Tabs anzeigen ?
        xleftmargin=0pt,
        framexleftmargin=0pt,
        framexrightmargin=0pt,
        framexbottommargin=0pt,
        %backgroundcolor=\color{lightgray},
        showstringspaces=false      % Leerzeichen in Strings anzeigen ?        
}
\lstloadlanguages{% Check Dokumentation for further languages ...
         %[Visual]Basic
         %Pascal
         %C
         %C++
         XML,
         HTML,
         Java,
         PHP
}

\hypersetup{
	pdfauthor = {Niels Doorn},
	pdftitle = {HTML en CSS Cheatsheet},
	pdfsubject = {HTML, CSS},
	pdfkeywords = {HTML,css,cheatsheet},
	pdfcreator = {NielsDoorn/RocVanTwente}
}

\usepackage{lastpage}
\usepackage{fancyhdr}
\pagestyle{fancy}
\rhead{}
\lhead{}
\chead{}
\lfoot{Niels Doorn \copyright~2012}
\cfoot{\url{http://www.nielsdoorn.nl}}
\rfoot{\thepage\ van \pageref{LastPage}}
\renewcommand{\headrulewidth}{0pt}
\renewcommand{\footrulewidth}{0pt}


\usepackage{pst-barcode}

\begin{document}

\ifpdf
\DeclareGraphicsExtensions{.pdf, .jpg, .tif}
\else
\DeclareGraphicsExtensions{.eps, .jpg}
\fi

\begin{multicols*}{4}

\section*{HTML \& CSS Cheatsheets}
Deze cheatsheet bevat de meest belangrijke HTML tags en CSS eigenschappen voor het vak webdesign in het eerste jaar.

\section*{HTML}
HTML vormt de structuur en de inhoud van een pagina en bestaat uit tags die door de browser worden begrepen. Iedere tag heeft zo z'n eigen specifieke eigenschappen en functionaliteit. Een aantal tags komen altijd voor in een pagina en sommige tags gebruik je alleen als je ze nodig hebt.
\subsection*{Minimale HTML pagina}
Ieder HTML pagina moet voldoen aan een bepaalde structuur. Hieronder een minimale pagina in HTML.
\lstinputlisting{code/minimal.html}
HTML tag: HTML document\\
HEAD tag: Informatie over de pagina\\
BODY tag: Inhoud van de pagina

\subsection*{Een CSS stylesheet koppelen}
Plaats in de HEAD:
\begin{lstlisting}[language=HTML]
	<link href="style.css" type="text/css" rel="stylesheet" />
\end{lstlisting}
\noindent Of:
\begin{lstlisting}[language=HTML]
<style type="text/css">
@import url("style.css");
</style>
\end{lstlisting}
\noindent Beiden voegen een stylesheet toe aan de pagina. Gebruik \underline{nooit} inline CSS maar \underline{altijd} een aparte stylesheet.

\subsection*{Commentaar}
Commentaar in HTML:
\begin{lstlisting}[language=HTML]
<!--
commentaar
-->
\end{lstlisting}

\subsection*{Tekst}
Headings:
\begin{lstlisting}[language=HTML]
<h1>Titel niveau 1</h1>
<h2>Titel niveau 2</h2>
...
<h6>Titel niveau 6</h6>
\end{lstlisting}

\noindent Tekstopmaak:
\begin{lstlisting}[language=HTML]
<b>Vetgedruk</b>
<i>Schuingedrukt</i>
<strong>Belangrijk</strong>
<em>Benadrukken</em>
E=MC<sup>2</sup> superscript
CO<sub>2</sub> subscript
<blockquote>Lang citaat</blockquote>
<q>Kort citaat</q> (werkt niet in IE)
\end{lstlisting}

\noindent Tekstindeling:
\begin{lstlisting}[language=HTML]
<p>Paragraaf</p>
<br /> Nieuwe regel
<hr /> Horizontale lijn
\end{lstlisting}

\subsection*{Pagina-indeling}
Div'jes gebruik je om de inhoud van je pagina op te delen in logische gebieden. Vervolgens kun je die met CSS een plaats, afmeting en andere eigenschappen geven. 
\begin{lstlisting}[language=HTML]
<div>Een gebied op een pagina, bijvoorbeeld een kolom</div>
\end{lstlisting}

\subsection*{Afbeeldingen}
Een afbeelding in HTML het src attribuut verwijst naar het bestand wat je wilt laten zien.
\begin{lstlisting}[language=HTML]
<img src="plaatje.jpg" />
\end{lstlisting}

\subsection*{Links}
Linkjes werken op de volgende manier. Eerst geef je in de A tag aan naar welke pagina je wilt linken met het href attribuut. Daarna plaats je tussen de begin en eind tag tekst of een afbeelding waar je op wilt klikken. In de onderstaande code is dat `Contactpagina'.

\begin{lstlisting}[language=HTML]
<a href="contact.html">Contactpagina</a>
\end{lstlisting}

\subsection*{Lijstjes}
In HTML zijn er twee belangrijke soorten lijstjes. Lijstjes die genummerd zijn (ge-ordend) en lijstjes met bullets (ongeordend).

\subsubsection*{Ongeordende lijsten}
Unordered list (boodschappenlijstje): 
\begin{lstlisting}[language=HTML]
<ul>
	<li>Banaan</li>
	<li>Aardbei</li>
	<li>Appel</li>
</ul>
\end{lstlisting}

\subsubsection*{Genummerde lijsten}
Een ordered list (grootste steden ter wereld):
\begin{lstlisting}[language=HTML]
<ol>
	<li>Shanghai</li>
	<li>Bombay</li>
	<li>Karachi</li>
	<li>Istanboel</li>
	<li>Delhi</li>
</ol>
\end{lstlisting}

\subsubsection*{Definitie lijsten}
Minder vaak gebruikt maar ook handig is de lijst met definities:
\begin{lstlisting}[language=HTML]
<dl>
	<dt>Koffie</dt>
		<dd>Het zwarte goud, onmisbaar voor goed functioneren docenten</dd>
	<dt>Melk</dt>
		<dd>De witte motor</dd>
</dl>
\end{lstlisting}
\noindent De DT tag geeft een woord waarvan je de definitie wilt geven (koffie). De daarop volgende DD tag geeft de definitie.

\subsubsection*{Lijsten met meeerdere niveaus}
Je kunt verschillende niveaus aanbrengen in lijstjes.

\begin{lstlisting}[language=HTML]
<ul>
  <li>Koffie</li>
  <li>Thee
    <ul>
    <li>Zwarte thee</li>
    <li>Groene thee</li>
    </ul>
  </li>
  <li>Melk</li>
</ul>
\end{lstlisting}

\subsection*{Handige websites}
Er zijn veel sites met informatie over HTML, deze twee worden veel gebruikt.
\begin{itemize}
	\item Lorem Ipsum generator \url{http://bit.ly/3Ukgp} 
	\item W3Schools \url{http://bit.ly/PkhA}
\end{itemize}

\clearpage

\section*{CSS}
Cascading Stylesheets (CSS) gebruik je voor het vormgeven van de website. 

\subsection*{Begin van een CSS file}
\begin{lstlisting}
@charset "utf-8";
\end{lstlisting}

\subsection*{Commentaar in CSS}
\begin{lstlisting}
/* commentaar */
\end{lstlisting}

\subsection*{CSS selectoren}
In CSS selecteer je de HTML elementen die je wilt vormgeven en vervolgens geef je aan hoe die elementen vormgegeven moeten worden.

\noindent Selecteren op TAG
\begin{lstlisting}
body {
}
div {
}
\end{lstlisting}

\noindent Selecteren op class (een class mag meerdere keren voorkomen op een HTML pagina)
\begin{lstlisting}
.menu-item {
}
\end{lstlisting}
\noindent Bijbehorende HTML code:
\begin{lstlisting}
<div class="menu-item"></div>
\end{lstlisting}

\noindent Selecteren op id (een id mag slechts een keer voorkomen op een pagina)
\begin{lstlisting}
#header {
}
\end{lstlisting}
\noindent Bijbehorende HTML code:
\begin{lstlisting}
<div id="header"></div>
\end{lstlisting}

\noindent Selecteren op type en id (een div met het id header)
\begin{lstlisting}
div#header {
}
\end{lstlisting}

\noindent Selecteren op type en class (alle divs met de klasse menu-item)
\begin{lstlisting}
div.menu-item {
}
\end{lstlisting}

\subsection*{Kleuren}
Achtergrondkleur, font kleur
\begin{lstlisting}
#inhoud {
	background-color: #0C9;
	color: white;
}
\end{lstlisting}

\subsection*{Padding}
De ruimte die aan de \underline{binnenkant} wordt vrijgehouden.
\begin{lstlisting}
#inhoud {
	padding-top: 10px; 
	/* alleen bovenaan 10px */
}
\end{lstlisting}
\noindent Of:
\begin{lstlisting}
#inhoud {
	padding: 10px; 
	/* rondom 10px */
}
\end{lstlisting}
\noindent Of:
\begin{lstlisting}
#inhoud {
	padding: 10px 0px 15px 0px; 
	/* top, right, bottom, left */
}
\end{lstlisting}

\subsection*{Margin}
De ruimte die aan de \underline{buitenkant} wordt vrijgehouden. Zie padding.
\begin{lstlisting}
	margin-top: 10px; 
	/* alleen bovenaan 10px */
	
	margin: 10px; 
	/* of rondom 10px */
	
	margin: 10px 0px 15px 0px; 
	/* of top, right, bottom, left */
\end{lstlisting}

\subsection*{Borders}
Sommige elementen, zoals DIV'jes kunnen ook worden voorzien van een border.
\begin{lstlisting}
div#sidebar {
	border-width: 1px;
	border-style: dashed;
	border-color: black;
}
\end{lstlisting}

\noindent Of geen borders:

\begin{lstlisting}
div#sidebar {
	border: none;
}
\end{lstlisting}

\subsection*{Afmetingen}
In pixels:
\begin{lstlisting}
#inhoud {
	width:300px;
	height: 400px;
}
\end{lstlisting}

\noindent In procenten:
\begin{lstlisting}
#inhoud {
	width:50%;
	height:20%;
}
\end{lstlisting}

\subsection*{Positionering}
Float gebruik je o.a. om een DIV naast een andere DIV te plaatsen
\begin{lstlisting}
#inhoud {
	float:left;
}
\end{lstlisting}
\noindent Of:
\begin{lstlisting}
#inhoud {
	float:right;
}
\end{lstlisting}

\subsection*{Kolommen omhoog pushen}
Met een clear both krijg je een div onder de andere divs.
\begin{lstlisting}
#footer {
	clear:both;
}
\end{lstlisting}

\subsection*{Een DIV centreren}
\begin{lstlisting}
div#container {
	width:850px;
	min-heigth:720px;
	margin-left:auto;
	margin-right:auto;
}
\end{lstlisting}

\subsection*{Achtergrondafbeelding}
\begin{lstlisting}
div#header {
	height:140px;
	background-image:url(achtergrond.jpg);
}
\end{lstlisting}

\subsection*{UL als menu stylen}
HTML met een UL als menu:
\begin{lstlisting}
<div id="menu"> 
  <ul>
	<li class="selected"><a href="index.html">Home</a></li>
    <li><a href="info.html">Info</a></li>
    <li><a href="contact.html">Contact</a></li>
  </ul>
</div>
\end{lstlisting}
CSS style:
\lstinputlisting{code/menu.css}

\end{multicols*}
\end{document}